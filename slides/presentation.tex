\documentclass{beamer}
\usepackage[utf8]{inputenc}
\usetheme{Singapore}


%\title{Apresentação  ME}
%\subtitle{Delineamento e Otimização de Helicópteros de Papel }
%\author{Leonardo Uchoa, Hugo Calegari, Lara Acrani}
%\institute{Universidade Estadual de Campinas}
%\date{ }

%\titlegraphic{\includegraphics[scale=0.2]{logo_unicamp}}

\begin{document}

\begin{frame}[plain]
    \title{Apresentação ME623}
    \subtitle{Delineamento e Otimização de Helicópteros de Papel}
    \institute{Universidade Estadual de Campinas}
    \date{ }
    \maketitle
    \small
    \begin{tabular}[t]{@{}l@{\hspace{3pt}}p{.32\textwidth}@{}}
        Autores: & Leonardo Uchoa \\
        &  Hugo Calegari \\
        &  Lara Acrani
    \end{tabular}%

    \begin{figure}
        \includegraphics[scale=0.2]{logo_unicamp}
    \end{figure}
\end{frame}

\begin{frame}
    \frametitle{Sumário}
        \tableofcontents
\end{frame}

\section{Introdução}
    \begin{frame}
        \frametitle{Introdução}
        Pesquisa sobre redução de tempo de vôo para helicópteros feitos com papel no intuito de, futuramente, ter um modelo base relevante -ou seja, utilizando princípios bem fundamentados de sólida estrutura matemática- para replicação em massa.
    \end{frame}

\section{Objetivo}
    \begin{frame}
        \frametitle{Objetivo}
        O objetivo desta pesquisa é encontrar as especificações ótimas de maneira que, dentro um padrão, façam com que um Helicóptero de Papel, com tais configurações, tenha o maior tempo de vôo possível em relação as demais características.
    \end{frame}

\section{Metodologia}
    \begin{frame}
        \frametitle{Metodologia}

        \begin{itemize}
            \item As técnicas empregadas aqui seguem a metodologia estatística de Delineamento de Experimentos. Tal escolha é motivada pela sua capacidade de direcionar/apontar quais são os níves ótimos para cada combinação de fatores, o que utilizado como "guia exploratório" desta pesquisa.

            \item Para cada um dos seis fatores, serão considerados dois níveis, o que configura um experimento fatorial com um total de $2^6$ possíveis combinações.

        \end{itemize}
    \end{frame}

    \subsection{Fatores}
        \begin{frame}
            \frametitle{Metodologia}
                \framesubtitle{Fatores}
                    Os fatores a serem analisados serão :

                        \begin{enumerate}
                            \item Formato da asa
                            \item Comprimento da asa
                            \item Largura da asa
                            \item Material da asa
                            \item Peso do helicóptero
                            \item peso do eixo
                        \end{enumerate}

        \end{frame}

    \subsection{Níveis}
            \begin{frame}
                \framesubtitle{Níveis}
                    \frametitle{Metodologia}

                        Já os níveis, respectivos aos fatores, serão
                            \begin{itemize}
                                \item
                                \item
                                \item
                                \item
                                \item
                                \item
                            \end{itemize}
        \end{frame}

    \subsection{Hipóteses}
         \begin{frame}
            \frametitle{Metodologia}
               \framesubtitle{Hipóteses}
                  A hipótese aqui assumida seguirá a equação
                     \begin{equation}
                        y_{ij}= \mu + \tau_i + \beta_j + {\epsilon_i}_j
                        \label{eq:equacao definidora}
                     \end{equation}
                  de forma que \large{$\epsilon$} segue uma distribuição normal com média 0 e variância $\sigma$, $\mu$ é um valor comum a todos os os efeitos, $\tau_i$ é um efeito comum a entre fatores e $\beta_j$ um efeito relativo entre níveis.
         \end{frame}

      \begin{frame}
         \frametitle{Metodologia}
            \framesubtitle{Hipóteses}
               \begin{itemize}
                  \item Para tal equação teremos, como proposto, um total de $\phi^k$ combinações entre características e tratamentos, onde $\phi$ é o número de níveis e k, a quantidade de fatores. Neste experimento, $k$ e $\phi$ assumem, respectivamente os valores 2 e 6.

                  \item Quanto à aleatorização, a ordem com que os helicóteros serão soltos será completamente aleatória, com o intuito de reduzir efeitos como, por exemplo, treinamento/condicionamento do operador que irá soltar o helicóptero.

               \end{itemize}
      \end{frame}

      \begin{frame}
         \frametitle{Metodologia}
            \framesubtitle{Hipóteses}
               \begin{itemize}
                  \item A maneira para determinar a ordem de descida dos protótipos será realizando uma permutação randômica de todos os $2^6$ protótipos no software $R$.

                  \item Também ocorrerá aleatorização entre os operadores que iram soltar o helicóptero para, novamente, tentar amenizar o efeito de condicionamento do operador. Já a escolha se dará ao utilizarmos um simulador de distribuição multinomial com três eventos equiprováveis, ou seja, 1/3 de probabilidade para cada operador.
               \end{itemize}
      \end{frame}

\section{Modelos}
    \begin{frame}
        \frametitle{Metodologia}
            \framesubtitle{Modelos}
        \begin{figure}
            %\label{fig:modelo tipo 1}
            %\includegraphics[scale=escala]{prototipo_aqui}
            %\caption{Modelo tipo I}
        \end{figure}
    \end{frame}




\end{document}
