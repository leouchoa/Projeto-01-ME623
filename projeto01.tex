\documentclass[10pt]{article}
\usepackage{geometry}
\usepackage{amsmath, amssymb, amsfonts, amsthm}
\usepackage[utf8]{inputenc}

\begin{document}
\title{Delineamento e Pesquisa de Protótipos de Helicóptero de Papel
                  Relatório Parcial}
\author{Leonardo Uchoa, Lara Acrani e Hugo Calegari.}

\date{ }
\maketitle

\cleardoublepage

\tableofcontents

\section{Introdução}

Uma brincadeira muito famosa que geralmente faz parte da infância de muitas crianças é criar/dobrar um modelo com a aerodinâmica de um avião, utilizando papel (geralmente folha de caderno), e arremessar-lo em um direção "qualquer".

Uma outra brincadeira, não tão famosa -mas igualmente divertida e interessante-, é , após construir um helicótero com papel e caneta,  soltá-lo e observar sua queda .Entretanto, na época em que a brincadeira é feita -infância-, geralmente não é mostrado a importância -via técnicas didáticas- noções de medição. Logo, a maneira abtualmente utilizada para comprovar qual era o melhor modelo era : cada criança construia o seu próprio brinquedo e o soltava de uma altura relativamente alta, cronometrando o tempo de queda .

Esta pesquisa  irá estudar a segunda brincadeira citada de uma maneira técnica e controlada, de forma que os efeitos e as interações entre os fatores/configurações (como tamanho da hélice, tipo de papel e vários outros) sejam realçados, nos guiando para o modelo que garanta o maior tempo de voo possível.

\section{Objetivo}
O objetivo deste projeto é, ao utilizar a técnica de Delineamento de Experimentos, estudar como as diversas especificações, que definem um helicóptero de papel, interagem entre sí e consequentemente encontrar a configuração  ótima de tempo de voo, ou seja, achar combinação entre fatores que forneça o maior tempo de voo possível dadas as especificações.

\section{Metodologia}


A metodologia que será empregada neste estudo será a de Delineamento de Experimentos.  O motivo desta escolha é justificado pela interação entre simplicidade da aplicação e sua habilidade de gerar  resultados contundentes e objetivos. Não obstante, os efeitos/tratamentos/fatores a serem estudados aqui serão

\begin{enumerate}
\item Formato da asa
\item  Comprimento da asa
\item Largura da asa
\item Material da asa
\item Peso do helicóptero
\item Rotação (ao iniciar o processo de queda livre) em direção fixa.
\item peso do eixo
\item enrijecimento da asa
\end{enumerate}

Outras características, como ângulo da asa, formato do eixo e comprimento do eixo serão mantido fixos para este experimento.

O local de teste será o segundo piso do Instituto de Matemática, Estatistica e Computação Científica, IMECC, pois é um lugar fechado, o que é favorável ao experimento, já que ameniza o efeito ventania e outros possíveis efeitos ruído. Adicionalmente, a maneira como o objeto será solto irá variar, uma vez que o efeito do fator rotação será avaliado e ordem de com que testaremos os efeitos será aleatória.

Partindo para a metodologia estatística, a estratégia chave utilizada para guiar este experimento será o delineamento fatorial $$\phi^k$$ onde $\phi$ é o número de níveis e k ,a quantidade de fatores. Entretanto, para podermos utilizar está técnica e consequentemente termos um resultado passível de conclusões, devemos supor algumas hipóteses e uma equação que relacione os efeitos com a média, que é um parâmetro muito importante em análise estatística. A equação será $$y_{ijk}= \mu + \tau_i + \beta_j + {\epsilon_i}_j,$$  onde  $\epsilon$ segue uma distribuição normal com média 0 e variância $\sigma$, $\mu$ é um valor comum a todos os os efeitos, $\tau_i$ é um efeito

Ao empregar este método ao modelo com dois níveis e sete fatores, temos um total de $2^{7}$ possíveis combinações de fatores.


\section{Bibliografia}


\end{document}
